% Options for packages loaded elsewhere
\PassOptionsToPackage{unicode}{hyperref}
\PassOptionsToPackage{hyphens}{url}
%
\documentclass[
]{article}
\usepackage{amsmath,amssymb}
\usepackage{lmodern}
\usepackage{iftex}
\ifPDFTeX
  \usepackage[T1]{fontenc}
  \usepackage[utf8]{inputenc}
  \usepackage{textcomp} % provide euro and other symbols
\else % if luatex or xetex
  \usepackage{unicode-math}
  \defaultfontfeatures{Scale=MatchLowercase}
  \defaultfontfeatures[\rmfamily]{Ligatures=TeX,Scale=1}
\fi
% Use upquote if available, for straight quotes in verbatim environments
\IfFileExists{upquote.sty}{\usepackage{upquote}}{}
\IfFileExists{microtype.sty}{% use microtype if available
  \usepackage[]{microtype}
  \UseMicrotypeSet[protrusion]{basicmath} % disable protrusion for tt fonts
}{}
\makeatletter
\@ifundefined{KOMAClassName}{% if non-KOMA class
  \IfFileExists{parskip.sty}{%
    \usepackage{parskip}
  }{% else
    \setlength{\parindent}{0pt}
    \setlength{\parskip}{6pt plus 2pt minus 1pt}}
}{% if KOMA class
  \KOMAoptions{parskip=half}}
\makeatother
\usepackage{xcolor}
\usepackage[margin=1in]{geometry}
\usepackage{color}
\usepackage{fancyvrb}
\newcommand{\VerbBar}{|}
\newcommand{\VERB}{\Verb[commandchars=\\\{\}]}
\DefineVerbatimEnvironment{Highlighting}{Verbatim}{commandchars=\\\{\}}
% Add ',fontsize=\small' for more characters per line
\usepackage{framed}
\definecolor{shadecolor}{RGB}{248,248,248}
\newenvironment{Shaded}{\begin{snugshade}}{\end{snugshade}}
\newcommand{\AlertTok}[1]{\textcolor[rgb]{0.94,0.16,0.16}{#1}}
\newcommand{\AnnotationTok}[1]{\textcolor[rgb]{0.56,0.35,0.01}{\textbf{\textit{#1}}}}
\newcommand{\AttributeTok}[1]{\textcolor[rgb]{0.77,0.63,0.00}{#1}}
\newcommand{\BaseNTok}[1]{\textcolor[rgb]{0.00,0.00,0.81}{#1}}
\newcommand{\BuiltInTok}[1]{#1}
\newcommand{\CharTok}[1]{\textcolor[rgb]{0.31,0.60,0.02}{#1}}
\newcommand{\CommentTok}[1]{\textcolor[rgb]{0.56,0.35,0.01}{\textit{#1}}}
\newcommand{\CommentVarTok}[1]{\textcolor[rgb]{0.56,0.35,0.01}{\textbf{\textit{#1}}}}
\newcommand{\ConstantTok}[1]{\textcolor[rgb]{0.00,0.00,0.00}{#1}}
\newcommand{\ControlFlowTok}[1]{\textcolor[rgb]{0.13,0.29,0.53}{\textbf{#1}}}
\newcommand{\DataTypeTok}[1]{\textcolor[rgb]{0.13,0.29,0.53}{#1}}
\newcommand{\DecValTok}[1]{\textcolor[rgb]{0.00,0.00,0.81}{#1}}
\newcommand{\DocumentationTok}[1]{\textcolor[rgb]{0.56,0.35,0.01}{\textbf{\textit{#1}}}}
\newcommand{\ErrorTok}[1]{\textcolor[rgb]{0.64,0.00,0.00}{\textbf{#1}}}
\newcommand{\ExtensionTok}[1]{#1}
\newcommand{\FloatTok}[1]{\textcolor[rgb]{0.00,0.00,0.81}{#1}}
\newcommand{\FunctionTok}[1]{\textcolor[rgb]{0.00,0.00,0.00}{#1}}
\newcommand{\ImportTok}[1]{#1}
\newcommand{\InformationTok}[1]{\textcolor[rgb]{0.56,0.35,0.01}{\textbf{\textit{#1}}}}
\newcommand{\KeywordTok}[1]{\textcolor[rgb]{0.13,0.29,0.53}{\textbf{#1}}}
\newcommand{\NormalTok}[1]{#1}
\newcommand{\OperatorTok}[1]{\textcolor[rgb]{0.81,0.36,0.00}{\textbf{#1}}}
\newcommand{\OtherTok}[1]{\textcolor[rgb]{0.56,0.35,0.01}{#1}}
\newcommand{\PreprocessorTok}[1]{\textcolor[rgb]{0.56,0.35,0.01}{\textit{#1}}}
\newcommand{\RegionMarkerTok}[1]{#1}
\newcommand{\SpecialCharTok}[1]{\textcolor[rgb]{0.00,0.00,0.00}{#1}}
\newcommand{\SpecialStringTok}[1]{\textcolor[rgb]{0.31,0.60,0.02}{#1}}
\newcommand{\StringTok}[1]{\textcolor[rgb]{0.31,0.60,0.02}{#1}}
\newcommand{\VariableTok}[1]{\textcolor[rgb]{0.00,0.00,0.00}{#1}}
\newcommand{\VerbatimStringTok}[1]{\textcolor[rgb]{0.31,0.60,0.02}{#1}}
\newcommand{\WarningTok}[1]{\textcolor[rgb]{0.56,0.35,0.01}{\textbf{\textit{#1}}}}
\usepackage{graphicx}
\makeatletter
\def\maxwidth{\ifdim\Gin@nat@width>\linewidth\linewidth\else\Gin@nat@width\fi}
\def\maxheight{\ifdim\Gin@nat@height>\textheight\textheight\else\Gin@nat@height\fi}
\makeatother
% Scale images if necessary, so that they will not overflow the page
% margins by default, and it is still possible to overwrite the defaults
% using explicit options in \includegraphics[width, height, ...]{}
\setkeys{Gin}{width=\maxwidth,height=\maxheight,keepaspectratio}
% Set default figure placement to htbp
\makeatletter
\def\fps@figure{htbp}
\makeatother
\setlength{\emergencystretch}{3em} % prevent overfull lines
\providecommand{\tightlist}{%
  \setlength{\itemsep}{0pt}\setlength{\parskip}{0pt}}
\setcounter{secnumdepth}{-\maxdimen} % remove section numbering
\ifLuaTeX
  \usepackage{selnolig}  % disable illegal ligatures
\fi
\IfFileExists{bookmark.sty}{\usepackage{bookmark}}{\usepackage{hyperref}}
\IfFileExists{xurl.sty}{\usepackage{xurl}}{} % add URL line breaks if available
\urlstyle{same} % disable monospaced font for URLs
\hypersetup{
  pdftitle={HDT2},
  pdfauthor={Elean Rivas, Javier Alvarez},
  hidelinks,
  pdfcreator={LaTeX via pandoc}}

\title{HDT2}
\author{Elean Rivas, Javier Alvarez}
\date{2023-02-16}

\begin{document}
\maketitle

\#\#\#Universidad del Valle de Guatemala \#\#\#Mineria de datos
\#\#\#Elean Rivas - 19062 \#\#\#Javier Alarez - 18051 \#\#\#Elean Rivas
- 19062

\begin{Shaded}
\begin{Highlighting}[]
\NormalTok{peliculas }\OtherTok{\textless{}{-}} \FunctionTok{read.csv}\NormalTok{(}\StringTok{"movies.csv"}\NormalTok{)}
\end{Highlighting}
\end{Shaded}

\#\#1. Haga el preprocesamiento del dataset, explique qué variables no
aportan información a la generación de grupos y por qué. Describa con
qué variables calculará los grupos.

\begin{Shaded}
\begin{Highlighting}[]
\NormalTok{variables }\OtherTok{\textless{}{-}} \FunctionTok{c}\NormalTok{(}\StringTok{"original\_title"}\NormalTok{, }\StringTok{"originalLanguage"}\NormalTok{, }\StringTok{"homePage"}\NormalTok{, }\StringTok{"video"}\NormalTok{, }\StringTok{"actorsCharacter"}\NormalTok{)}
\NormalTok{DF.variable }\OtherTok{\textless{}{-}} \FunctionTok{data.frame}\NormalTok{(variables)}
\FunctionTok{print}\NormalTok{(DF.variable)}
\end{Highlighting}
\end{Shaded}

\begin{verbatim}
##          variables
## 1   original_title
## 2 originalLanguage
## 3         homePage
## 4            video
## 5  actorsCharacter
\end{verbatim}

Estas variables son (en nuestra opinion) variables que no ayudan con la
generacion de los grupos ya que tienen caracteristicas propias que no se
relacionan con las demas y/o contienen informacion no usable.

\#\#2. Analice la tendencia al agrupamiento usando el estadístico de
Hopkings y la VAT (Visual Assessment of cluster Tendency). Discuta sus
resultados e impresiones.

\begin{Shaded}
\begin{Highlighting}[]
\FunctionTok{library}\NormalTok{(hopkins)}
\NormalTok{peliculas }\OtherTok{\textless{}{-}}\NormalTok{ peliculas[}\FunctionTok{complete.cases}\NormalTok{(peliculas),]}
\NormalTok{popularity }\OtherTok{\textless{}{-}}\NormalTok{ peliculas[, }\StringTok{\textquotesingle{}popularity\textquotesingle{}}\NormalTok{]}
\NormalTok{budget }\OtherTok{\textless{}{-}}\NormalTok{ peliculas[, }\StringTok{\textquotesingle{}budget\textquotesingle{}}\NormalTok{]}
\NormalTok{revenue }\OtherTok{\textless{}{-}}\NormalTok{ peliculas[,}\StringTok{\textquotesingle{}revenue\textquotesingle{}}\NormalTok{]}
\NormalTok{runtime }\OtherTok{\textless{}{-}}\NormalTok{ peliculas[,}\StringTok{\textquotesingle{}runtime\textquotesingle{}}\NormalTok{]}
\NormalTok{voteCount }\OtherTok{\textless{}{-}}\NormalTok{ peliculas[,}\StringTok{\textquotesingle{}voteCount\textquotesingle{}}\NormalTok{]}
\NormalTok{normd }\OtherTok{\textless{}{-}} \FunctionTok{data.frame}\NormalTok{(popularity,budget,revenue,runtime, voteCount )}
\NormalTok{clustering }\OtherTok{\textless{}{-}} \FunctionTok{scale}\NormalTok{(normd)}

\FunctionTok{hopkins}\NormalTok{(clustering)}
\end{Highlighting}
\end{Shaded}

\begin{verbatim}
## [1] 0.9949134
\end{verbatim}

\begin{Shaded}
\begin{Highlighting}[]
\NormalTok{dataDist }\OtherTok{\textless{}{-}} \FunctionTok{dist}\NormalTok{(clustering)}
\end{Highlighting}
\end{Shaded}

\begin{verbatim}
## Welcome! Want to learn more? See two factoextra-related books at https://goo.gl/ve3WBa
\end{verbatim}

\includegraphics{hdt2_files/figure-latex/unnamed-chunk-4-1.pdf}

Podemos observar que el valor que retorna la funcion de hopkins esta
bastante alejado de 0.5, por lo podemos decir que los datos recopilados
no son aleatorios.

\#\#3. Determine cuál es el número de grupos a formar más adecuado para
los datos que está trabajando. Haga una gráfica de codo y explique la
razón de la elección de la cantidad de clústeres con la que trabajará.

\begin{Shaded}
\begin{Highlighting}[]
\NormalTok{wss }\OtherTok{=} \DecValTok{0}
\ControlFlowTok{for}\NormalTok{ (i }\ControlFlowTok{in} \DecValTok{1}\SpecialCharTok{:}\DecValTok{10}\NormalTok{)}
\NormalTok{  wss[i] }\OtherTok{\textless{}{-}} \FunctionTok{sum}\NormalTok{(}\FunctionTok{kmeans}\NormalTok{(clustering[,}\DecValTok{1}\SpecialCharTok{:}\DecValTok{5}\NormalTok{], }\AttributeTok{centers=}\NormalTok{i)}\SpecialCharTok{$}\NormalTok{withinss)}
\FunctionTok{plot}\NormalTok{(}\DecValTok{1}\SpecialCharTok{:}\DecValTok{10}\NormalTok{, wss, }\AttributeTok{type=}\StringTok{"b"}\NormalTok{, }\AttributeTok{xlab=}\StringTok{"Numero de Clusters"}\NormalTok{,  }\AttributeTok{ylab=}\StringTok{"WSS"}\NormalTok{)}
\end{Highlighting}
\end{Shaded}

\includegraphics{hdt2_files/figure-latex/unnamed-chunk-5-1.pdf}

Basado en el resultado, podemos decir que el número de clusters óptimo
para analizar los datos es 6.

\#\#4. Utilice los algoritmos k-medias y clustering jerárquico para
agrupar. Compare los resultados generados por cada uno.

K-medias:
\includegraphics{hdt2_files/figure-latex/unnamed-chunk-6-1.pdf}
\includegraphics{hdt2_files/figure-latex/unnamed-chunk-6-2.pdf}
\includegraphics{hdt2_files/figure-latex/unnamed-chunk-6-3.pdf}

El resultado es muy cercano a 1, el cual es un resultado deseable.

clustering jerárquico

\includegraphics{hdt2_files/figure-latex/unnamed-chunk-7-1.pdf}
\includegraphics{hdt2_files/figure-latex/unnamed-chunk-7-2.pdf}
\includegraphics{hdt2_files/figure-latex/unnamed-chunk-7-3.pdf}

Similar al resultado anterior, obtuvimos un resultado deseable, cercano
a 1

\#\#5 Determine la calidad del agrupamiento hecho por cada algoritmo con
el método de la silueta. Discuta los resultados.

K-mean En el caso del Kmean, por método de la siluete obtuvimos que la
primera agrupación se comporta de manera coherente, en el caso de la 2da
y 3ra muestra vemos unos valores atipicos, pero con un coeficiente
adecuado cercano a 1

Cluster jerarquico

Los clusters jerarquicos mostraron una consistencia en la data más que
aceptable sin mostrar valores atípicos muy altos o incongruentes

Mezcla de gaussiano

El caso de la mezcla gaussiana fue la que provoco mayor disparidad en
los datos, pues la cantidad de información atípica fue la más alta, esto
puede observarse en los dos primeros clusters, aunque el tercero muestra
una agrupación con gran coherencia casi acercandose al cero.

\#\#6 Interprete los grupos basado en el conocimiento que tiene de los
datos. Recuerde investigar las medidas de tendencia central de las
variables continuas y las tablas de frecuencia de las variables
categóricas pertenecientes a cada grupo. Identifique hallazgos
interesantes debido a las agrupaciones y describa para qué le podría
servir.

\begin{Shaded}
\begin{Highlighting}[]
\FunctionTok{mean}\NormalTok{(}\AttributeTok{x =}\NormalTok{ normd}\SpecialCharTok{$}\NormalTok{popularity, }\AttributeTok{na.rm =} \ConstantTok{TRUE}\NormalTok{)}
\end{Highlighting}
\end{Shaded}

\begin{verbatim}
## [1] 68.1854
\end{verbatim}

\begin{Shaded}
\begin{Highlighting}[]
\FunctionTok{mean}\NormalTok{(}\AttributeTok{x =}\NormalTok{ normd}\SpecialCharTok{$}\NormalTok{budget, }\AttributeTok{na.rm =} \ConstantTok{TRUE}\NormalTok{)}
\end{Highlighting}
\end{Shaded}

\begin{verbatim}
## [1] 24335422
\end{verbatim}

\begin{Shaded}
\begin{Highlighting}[]
\FunctionTok{mean}\NormalTok{(}\AttributeTok{x =}\NormalTok{ normd}\SpecialCharTok{$}\NormalTok{revenue, }\AttributeTok{na.rm =} \ConstantTok{TRUE}\NormalTok{)}
\end{Highlighting}
\end{Shaded}

\begin{verbatim}
## [1] 78593938
\end{verbatim}

\begin{Shaded}
\begin{Highlighting}[]
\FunctionTok{mean}\NormalTok{(}\AttributeTok{x =}\NormalTok{ normd}\SpecialCharTok{$}\NormalTok{runtime, }\AttributeTok{na.rm =} \ConstantTok{TRUE}\NormalTok{)}
\end{Highlighting}
\end{Shaded}

\begin{verbatim}
## [1] 103.2592
\end{verbatim}

\begin{Shaded}
\begin{Highlighting}[]
\FunctionTok{round}\NormalTok{(}\FunctionTok{mean}\NormalTok{(}\AttributeTok{x =}\NormalTok{ normd}\SpecialCharTok{$}\NormalTok{voteCount, }\AttributeTok{na.rm =} \ConstantTok{TRUE}\NormalTok{))}
\end{Highlighting}
\end{Shaded}

\begin{verbatim}
## [1] 1871
\end{verbatim}

\begin{Shaded}
\begin{Highlighting}[]
\NormalTok{tab }\OtherTok{\textless{}{-}} \FunctionTok{table}\NormalTok{(normd}\SpecialCharTok{$}\NormalTok{popularity)}
\FunctionTok{head}\NormalTok{(}\FunctionTok{sort}\NormalTok{(tab, }\AttributeTok{decreasing =} \ConstantTok{TRUE}\NormalTok{), }\AttributeTok{n =} \DecValTok{15}\NormalTok{)}
\end{Highlighting}
\end{Shaded}

\begin{verbatim}
## 
## 15.804 39.372   8.54  9.336   9.34  9.363  9.608   9.83 10.171 10.192 10.243 
##      3      3      2      2      2      2      2      2      2      2      2 
## 10.303 10.393 10.472 10.767 
##      2      2      2      2
\end{verbatim}

\begin{Shaded}
\begin{Highlighting}[]
\NormalTok{tab }\OtherTok{\textless{}{-}} \FunctionTok{table}\NormalTok{(normd}\SpecialCharTok{$}\NormalTok{budget)}
\FunctionTok{head}\NormalTok{(}\FunctionTok{sort}\NormalTok{(tab, }\AttributeTok{decreasing =} \ConstantTok{TRUE}\NormalTok{), }\AttributeTok{n =} \DecValTok{15}\NormalTok{)}
\end{Highlighting}
\end{Shaded}

\begin{verbatim}
## 
##         0  20000000  10000000  30000000  15000000  40000000  25000000  50000000 
##      1688        84        81        77        70        68        57        56 
##   5000000  35000000  60000000  12000000 100000000 150000000   4000000 
##        52        45        45        42        34        33        32
\end{verbatim}

\begin{Shaded}
\begin{Highlighting}[]
\NormalTok{tabla }\OtherTok{\textless{}{-}} \FunctionTok{table}\NormalTok{(normd}\SpecialCharTok{$}\NormalTok{revenue)}
\FunctionTok{head}\NormalTok{(}\FunctionTok{sort}\NormalTok{(tabla, }\AttributeTok{decreasing =} \ConstantTok{TRUE}\NormalTok{), }\AttributeTok{n =} \DecValTok{15}\NormalTok{)}
\end{Highlighting}
\end{Shaded}

\begin{verbatim}
## 
##        0    7e+06     5000    2e+06  3600000  4100000  1.4e+07  1.9e+07 
##     1558        3        2        2        2        2        2        2 
## 21200000  2.6e+07  2.7e+07    3e+07 34100000  4.3e+07 43300000 
##        2        2        2        2        2        2        2
\end{verbatim}

\begin{Shaded}
\begin{Highlighting}[]
\NormalTok{tabla }\OtherTok{\textless{}{-}} \FunctionTok{table}\NormalTok{(normd}\SpecialCharTok{$}\NormalTok{runtime)}
\FunctionTok{head}\NormalTok{(}\FunctionTok{sort}\NormalTok{(tabla, }\AttributeTok{decreasing =} \ConstantTok{TRUE}\NormalTok{), }\AttributeTok{n =} \DecValTok{15}\NormalTok{)}
\end{Highlighting}
\end{Shaded}

\begin{verbatim}
## 
##  90 100  97  93 102 110  92  95  94 107  96 108  98 109 104 
## 143 115  98  93  90  88  86  86  84  84  83  83  82  82  80
\end{verbatim}

\begin{Shaded}
\begin{Highlighting}[]
\NormalTok{tabla }\OtherTok{\textless{}{-}} \FunctionTok{table}\NormalTok{(normd}\SpecialCharTok{$}\NormalTok{voteCount)}
\FunctionTok{head}\NormalTok{(}\FunctionTok{sort}\NormalTok{(tabla, }\AttributeTok{decreasing =} \ConstantTok{TRUE}\NormalTok{), }\AttributeTok{n =} \DecValTok{15}\NormalTok{)}
\end{Highlighting}
\end{Shaded}

\begin{verbatim}
## 
##   4  18 179   6  13  52  65   3  34  46  57  90 215   1   2 
##  15  13  12  11  11  11  11  10  10  10  10  10  10   9   9
\end{verbatim}

\end{document}
